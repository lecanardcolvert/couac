\documentclass[letter,12pt]{exam}
\author{Jordan Gauthier (GAUJ25089201)\\
Alex Dufour (DUFA23059001)\\
Alexandre H. Bourdeau (HAMA12128907)\\
Philippe Soucy (SOUP17039601)}

\title{Planification du projet\\
Équipe : la marque Sans nom}
\date{}

\begin{document}

\maketitle

\section{Description}\label{description}

\begin{itemize}

\item Utilisateurs : passionnés de sports ou de loisirs. (18 ans + pour la
réservation)
\item Exemple d'entreprises qui offrent leurs services : municipalités, centre de sports, centres de loisirs, location d'activités

\item Client : Diallo, Abdoulaye Banire, Ph. D

\end{itemize}

\section{Fonctionnalités}\label{fonctionnalites}

\begin{enumerate}

\subsection{Premier Sprint {[} 18 février 2020
{]}}\label{premier-sprint-18-fevrier-2020}


\tightlist
\item
  Implémentation d'un gabarit du site web
\item
  Implémentation d'une page contact
\item
  Implémentation de création et modification de comptes utilisateur
  (particulier)

\subsection{Deuxième Sprint {[} 17 mars 2020
{]}}\label{deuxieme-sprint-17-mars-2020}

\item
  Plateforme d'inscription pour les entreprises afin d'offrir leurs
  services
\item
  Accès à une page d'accueil
\item
  Accès à une page utilisateur
\item
  Implémentation de création et modification de comptes administrateur
\item
  Implémentation d'un gestionnaire d'événements (création, inscription et
  affichage)
\item
  Ajout d'utilisateurs à une liste ``d'amis" (afin de lui envoyer un message)
\item
  Implémentation d'une messagerie privée entre utilisateurs


\subsection{Produit final {[} 21 avril 2020
{]}}\label{produit-final-21-avril-2020}


\item
  Modifier les informations/préférences de comptes utilisateur
\item
  Implémementation d'un chat entre utilisateurs
\item
  Ajout et affichage de services d'une entreprise
\item
  Implémentation de l'affichage des événements similaires
\item
  Implémentation d'une barre de recherche d'événements (par ville,
  sport, activité, etc.)
\item
  Affichage d'historique d'événements
\item
  Ajout d'un service de commerce (panier) en ligne afin de louer les équipements ou infrastructures
\item
  Affichage de tableaux statistiques
\end{enumerate}

\section{Technologies du projet}

\subsection{Technologies utilisées}

\begin{itemize}
\tightlist
\item
  HTML, CSS, JavaScript, Bootstrap

  Bootstrap a été retenu pour l'implémentation de l'interface du site web en raison de sa simplicité d'utilisation tout-en-un pour la conception frontend.
\item
  R, Shiny

  Shiny, un package de R, a été retenu pour son intégration du langage R, utile pour les statistiques, dans un dashboard web.
\item
  MySQL, ASP.NET Core

  MySQL et ASP.NET Core ont été retenus pour la gestion de la base de données et la conception backend.
\end{itemize}

\subsection{Outils de développement}

\begin{itemize}

\item
  RStudio

  RStudio a été retenu comme IDE pour la gestion des statistiques sur la fréquentation des événements à afficher sur la plateforme Web.

\item
  GitHub

  GitHub a été retenu pour la gestion des sources et des versions puisque la majorité des membres de l'équipe le connaît bien.

\item
  Slack

  Slack a été retenu pour la communication entre membres de l'équipe et l'échange de fichiers autrement que par GitHub.
\item
  Excel

  Excel a été retenu pour la gestion des semainiers. Toute autre suite compatible est aussi acceptée.

\end{itemize}

\subsection{Production de documentation}

\begin{itemize}

\item
  Markdown

  Markdown a été retenu pour sa simplicité d'utilisation afin de concevoir la documentation interne.
\item
  LaTex, Pandoc

  LaTex a été retenu pour sa compatibilité avec un gestionnaire de source tel que GitHub pour la conception de la documentation client. Pandoc sera utilisé pour convertir les documents Markdown en LaTex au besoin.

\end{itemize}

\section{Compétences des membres}

\noindent{\bf Alexandre H. Bourdeau}\\

Compétences actuelles :

\begin{itemize}
\item
  HTML, CSS

\item JavaScript

\item MySQL

\item PHP

\end{itemize}

\noindent{\bf Philippe Soucy}\\

Compétences actuelles :

\begin{itemize}
\item HTML

\item R

\item Linux/Unix/bash

\item Tests logiciels

\item Débogage

\item Développement orienté objet (Java)

\item LaTex

\end{itemize}


Compétences à acquérir :

\begin{itemize}

\item CSS, JavaScript, Bootstrap

\item Shiny

\item MySQL, ASP.NET Core

\end{itemize}

\end{document}
