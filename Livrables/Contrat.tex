
\documentclass[letter,12pt]{exam}

\usepackage{multicol}

\title{Contrat de développement logiciel\\}
\date{}

\begin{document}

\maketitle

Ce présent contrat de développement logiciel entre en vigueur le
\texttt{28\ janvier\ 2020}. Il est conclut entre les parties suivantes :

\vspace{0.5cm}
\noindent{\bf La marque Sans nom (développeurs)}\\
\large\textbf{\emph{La marque Sans nom (développeurs)}}

\begin{itemize}
\item[]  M. Dufour, Alex
\item[]  M. Bourdeau, Alexandre H.
\item[]  M. Gauthier, Jordan
\item[]  M. Soucy, Philippe
\end{itemize}

\vspace{0.5cm}
\noindent{\bf ET}\\

M. Diallo, Abdoulaye Baniré
Ph.D. (client)


\section{Titre du projet}

Le produit final se nomme \emph{Plateforme Web événementielle et interactive}.


\section{Livrables}

\subsection{Planification du projet}

Les développeurs s'engagent à fournir les spécifications de la
plateforme Web définissant entre autres les technologies utilisées pour
réaliser le projet, la compétence des membres, la planification des
tâches et la répartions des rôles.

\subsection{Semainier}

Les développeurs s'engagent à livrer un semainier hebdomadairement
comprenant les tâches à réaliser tout au long du projet et le nombre
d'heures accordées à chacune de celle-ci. Les développeurs devront
indiquer individuellement combien d'heures ils ont consacrées à une tâche
précise.

\subsection{Rapport de sprint}

À la fin des deux (2) premiers sprints, les développeurs s'engagent à fournir un
rapport de rendement couvrant l'entièreté de la période du dit sprint.

Le document rendu devra comporter l'identification du et de la personne
responsable, la description du mandat tel qu'entendu avec la personne
responsable, la description détaillée des activités planifiées et
réalisées à cette étape, l'identification des problèmes rencontrés et le
réajustement de la planification.

\subsection{Rapport final}

À la fin du projet, les développeurs s'engagent à livrer un rapport
final comprenant l'identification du projet et de la personne
responsable, la description du contrat tel qu'entendu avec la personne
responsable et les différentes modifications et ententes conclues, la
description détaillée des activités planifiées et réalisées,
l'identification des solutions apportées aux problèmes rencontrés,
l'identification des problèmes généraux rencontrés, l'évaluation du
nombre d'heures consacrées au projet en entier, l'évaluation du produit
fourni.

\section{Services}

Les développeurs s'engagent envers le client à fournir les services
suivants :

\subsection{Plateforme Web (description sommaire du
projet)}

Concevoir, développer et programmer une plateforme Web interactive de
création, d'inscription et de partage d'événements entre des
organisations et des particuliers. Les utilisateurs de la plateforme
pourront échanger, collaborer dans le but d'organiser collectivement des
événements. Des organisations telles que des centres sportifs ou de loisir
pourront utiliser la plateforme pour la location de leurs
infrastructures et équipements, et ainsi alimenter la banque de
ressources événementielles collective. La plateforme devra comprendre
minimalement les éléments suivant :

\vspace{0.5cm}
\noindent{\bf (Objectifs à atteindre)}\\

\begin{itemize}
\tightlist
\item[a)]
  Une interface graphique permettant la navigation entre différentes
  sections de la plateforme, incluant l'affichage de :

  \begin{itemize}
  \tightlist
  \item
    une page d'accueil
  \item
    une page contact
  \item
    une page de profil utilisateur
  \end{itemize}

\newpage
\item[b)]
  Un gestionnaire d'événements permettant la création, l'inscription et
  l'affichage des informations du dit événement incluant :

  \begin{itemize}
  \tightlist
  \item
    un système de recommandation d'événements
  \item
    un historique des événements passés
  \item
    un système d'analyse et d'affichage statistique sur la fréquentation
    des événements
  \end{itemize}
\item[c)]
  Gestion de comptes utilisateur

  \begin{itemize}
  \tightlist
  \item
    ajout d'utilisateurs
  \item
    modification des informations du profil
  \end{itemize}
\item[d)]
  Gestion de comptes entreprises
\item[e)]
  Une plateforme d'échange de messages privés (messagerie) entre
  utilisateurs
\item[f)]
  Une plateforme de clavardage entre utilisateurs
\item[g)]
  Des pages d'entreprise
\item[h)]
  Une barre de recherche d'événements
\item[i)]
  Un système d'achats en ligne permettant de louer équipements et
  infrastructures.
\end{itemize}

\section{Délai de fourniture des services\\ (livrables à travers les
trois
sprints)}

Les développeurs s'engagent à remettre les livrables selon l'échéancier
suivant :

\begin{itemize}
\tightlist
\item[a)]
  La planification du projet le \texttt{28\ janvier\ 2020}.
\item[b)]
  Le premier rapport de sprint le \texttt{18\ février\ 2020}. Ce rapport
  sera livré avec une présentation des fonctionnalités suivantes :

  \begin{itemize}
  \tightlist
  \item
    l'interface
  \item
    l'affichage de la page de contact
  \item
    la gestion des comptes utilisateur
  \end{itemize}
\item[c)]
  Le deuxième rapport de sprint le \texttt{17\ mars\ 2020}. Ce rapport
  sera livré avec une présentation des fonctionnalités suivantes :

  \begin{itemize}
  \tightlist
  \item
    la gestion des comptes entreprise et administrateur
  \item
    l'affichage de la page d'accueil et de profil utilisateur
  \item
    le gestionnaire d'événements
  \item
    la messagerie
  \end{itemize}
\item[d)]
  La présentation du projet le \texttt{17\ avril\ 2020}.
\item[e)]
  Le produit final et le rapport final le \texttt{21\ avril\ 2020}. Le
  produit sera livré avec les fonctionnalités suivantes (en plus de
  celles présentées dans la même section) :

  \begin{itemize}
  \tightlist
  \item
    la modification de profil utilisateur
  \item
    le module de clavardage
  \item
    la gestion de comptes entreprise
  \item
    la recommandation, la recherche et l'historique d'événements
  \item
    la plateforme d'achats en ligne
  \item
    l'analyse statistique de participation aux événements
  \end{itemize}
\end{itemize}

\section{Technologies utilisées}

\begin{multicols}{2}
\begin{itemize}
\tightlist
\item
  HTML
\item
  CSS
\item
  JavaScript
\item
  Bootstrap
\item
  R
\item
  Shiny
\item
  MySQL
\item
  ASP.NET Core
\item
  RStudio
\item
  GitHub
\item
  Slack
\item
  Excel
\item
  Markdown
\item
  LaTex
\item
  Pandoc
\end{itemize}
\end{multicols}

\vspace{3cm}

\section{Signature du client}

\vspace{1cm}

\begin{tabular}{@{}p{0.2cm}p{4in}@{}}
& \hrulefill \\
& M. Diallo, Abdoulaye Baniré, Ph.D. \\
\end{tabular}

\end{document}
